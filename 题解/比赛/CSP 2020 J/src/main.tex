\documentclass{article}
\usepackage{graphicx}
\usepackage{amsfonts}
\usepackage[utf8]{inputenc}
\usepackage{xeCJK}
\usepackage{amsmath}
\usepackage{bm}
\usepackage{minted}
\usepackage{hyperref}
\usepackage[dvipsnames]{xcolor}
\usepackage{parskip}
\usepackage{soul}
\usepackage{cancel}
\usepackage{ragged2e}
\usepackage[normalem]{ulem}
\usepackage[stable]{footmisc}

\title{CSP 2020 J 题解}
\author{张晴川@实验舱}


\begin{document}

\maketitle
\section{优秀的拆分}
\subsection*{大意}
给定正整数 $n$,将其拆分为若干不同的 $2$ 的\textbf{正整数}次幂之和或输出无解。
\subsection*{数据范围}
\begin{itemize}
\item $1\le n \le 10^7$
\end{itemize}
\subsection*{题解}
由于 $2$ 的正整数次幂一定是偶数,如果 $n$ 是奇数,直接输出无解。否则从大到小枚举 $2$ 的幂,如果 $n$ 在二进制下的的对应位是 $1$ 就输出。具体参考代码。
\subsection*{复杂度}
\begin{itemize}
\item 时间:$O(\log(n))$
\item 空间:$O(1) $
\end{itemize}
\subsection*{代码}
\inputminted[linenos,autogobble]{cpp}{T1.cpp}


\section{直播获奖}
\subsection*{大意}
有 $n$ 个人,第 $i$ 个人的分数为一个整数 $a_i$。给定获奖比例 $w$。假设只考虑前 $p$ 个人,那么获奖分数线为从高到低第 $\max(1,\lfloor p*w\% \rfloor)$ 个人的成绩。求出 $p = 1,2,\ldots,n$ 时的分数线。

\subsection*{数据范围}
\begin{itemize}
\item $1\le n\le 10^5$
\item $1\le w \le 99$
\item $0\le a_i \le 600$
\end{itemize}
\subsection*{题解}
考虑模拟题意,枚举 $p = 1,2,\ldots,n$。首先计算使用题中的公式计算出获奖人数,注意需要全部使用整数计算。

如果每次将 $p$ 个人的成绩重新排序然后得到分数线,那么只能得到过 $6/20$ 个点。但如果每次使用插入排序,(应当)可以通过 $n \le 2000$ 的测试点。

考虑到每个人的分数都是 $[0,600]$ 内的整数,我们只需要对每个成绩开一个桶,每次相同分数的人可以一起考虑。该做法复杂度为 $O(601n)$,可以通过。

\subsection*{复杂度}
\begin{itemize}
\item 时间:$O(601n)$
\item 空间:$O(601)$
\end{itemize}
\subsection*{代码}
\inputminted[linenos,autogobble]{cpp}{T2.cpp}

\section{表达式}
\subsection*{大意}
给定一个包含 $n$ 个布尔变量、与运算$\&$、或运算$|$以及取反运算 $!$ 的后缀表达式 $s$。每个变量出现恰好一次。给定每个变量的初始值。每次询问将一个变量取反,求原始表达式的真值。每次询问独立,即并不真的修改变量的值。

\subsection*{数据范围}
\begin{itemize}
\item $1\le |s| \le 10^6$
\item $1\le q\le 10^5 $
\item $2\le n \le 10^5$
\end{itemize}
\subsection*{题解}
首先使用一个栈将后缀表达式转为一棵表达式树,并记录每个变量对应的节点。

针对 $n \le 1000$ 的测试点,我们只需要每次修改对应变量的值,并DFS整棵表达式树重新求值即可,复杂度为 $O(n^2)$。

为了通过所有测试点,首先我们需要用DFS自下而上预处理出表达树上每棵子树的值。现在我们在每棵子树的根节点上开设两个变量,$dp[0/1]$,分别表示当前子树的值是 $0/1$ 时,整棵表达式树的值。但这应如何计算?

对于根节点,显然 $dp[0/1] = 0/1$。而对于非根节点,我们可根据其父亲的操作符类型分类讨论,$\texttt{cousin->val}$表示兄弟节点的初始值:

\begin{itemize}
    \item $\&$:那么 $\texttt{dp[v] = father->dp[v \& cousin->val]}$
    \item $|$:那么 $\texttt{dp[v] = father->dp[v | cousin->val]}$
    \item $!$:那么 $\texttt{dp[v] = father->dp[!v]}$
\end{itemize}

根据以上公式,我们可以自上而下地 $O(n)$ 计算出每个节点的 $dp$ 值。针对每次询问,只需要 $O(1)$ 查表回答即可。

\subsection*{复杂度}
\begin{itemize}
\item 时间:$O(n)$
\item 空间:$O(n)$
\end{itemize}
\subsection*{代码}
\inputminted[linenos,autogobble]{cpp}{T3.cpp}

\section{方格取数}
\subsection*{大意}
给定二维矩阵 $a[n][m]$。现在需要从左上角走到右下角,每次可以向上、下、右走一步,但不能重复经过同一个格子,求最大化走过的格子的权值和是多少。
\subsection*{数据范围}
\begin{itemize}
\item $1\le n,m\le 10^3$
\item $|a[i][j]|\le 10^4$
\end{itemize}
\subsection*{题解}
设 $dp[i][j]$ 为从起点走到 $a[i][j]$ 的最大权值和。

第一列的值是容易求的,因为从起点出发只有一种走法。

考虑已经求完了第 $j$ 列的 $dp$ 值,应该如何推出第 $j+1$ 列的值呢?

假设在计算 $dp[i][j+1]$,我们可以枚举是从第 $j$ 列的哪个格子向右走过来的,不妨设为 $i'$。这里有两种情况 $i'\le i$ 和 $i' \ge i$,先假设 $i'\le i$,另一种情况类似。那么可以得到:
$$
dp[i][j+1] = \min_{1\le i'\le i}(dp[i'][j] + \sum_{x = i'}^i a[x][j+1])
$$

根据这个式子暴力计算,可以通过 70\% 的数据。

但我们可以利用式子的特殊性质进行加速,注意到:
\begin{align}
    dp[i+1][j+1] 
    &= \min_{1\le i'\le i+1}(dp[i'][j] + \sum_{x = i'}^{i+1} a[x][j+1])\\
    &= \min_{1\le i'\le i+1}(dp[i'][j] + \sum_{x = i'}^{i} a[x][j+1]) + a[i+1][j+1]  \\
    &= \min(dp[i][j+1],dp[i+1][j]) + a[i+1][j+1]  
\end{align}

所以只需要从上面的格子 $O(1)$ 转移过来即可,对于 $i' \ge i$ 的情况也一样。于是对于每一列,只需要上下各扫一次即可完成计算,总复杂度 $O(nm)$,可以通过。

\subsection*{复杂度}
\begin{itemize}
\item 时间:$O(nm)$
\item 空间:$O(nm)$
\end{itemize}
\subsection*{代码}
\inputminted[linenos,autogobble]{cpp}{T4.cpp}




\end{document}
