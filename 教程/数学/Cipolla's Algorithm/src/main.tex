\documentclass{article}
\usepackage{graphicx}
\usepackage{amsfonts}
\usepackage[utf8]{inputenc}
\usepackage{xeCJK}
\usepackage{amsmath}
\usepackage{amsthm}
\usepackage{bm}
\usepackage{minted}
\usepackage{hyperref}
\usepackage[dvipsnames]{xcolor}
\usepackage{parskip}
\usepackage{soul}
\usepackage{amssymb}
\usepackage{cancel}
\usepackage{ragged2e}
\usepackage[normalem]{ulem}
\usepackage[stable]{footmisc}
\newtheorem*{lemma*}{Lemma}


\title{Cipolla's algorithm\footnote{更多内容请访问:\url{https://github.com/SamZhangQingChuan/Editorials}}}
\author{张晴川\\\href{mailto:qzha536@aucklanduni.ac.nz}{\texttt{qzha536@aucklanduni.ac.nz}}}

\begin{document}
\maketitle

\section{问题}
\begin{center}
给定素数$p$ 和整数$n$,在 $\mathbb{F}_p$ 上求平方根 $\sqrt{n}$,$p$是素数。
\end{center}
\section{核心思路}
我们在 $\mathbb{F}_p$ 的二次扩张上求平方根,如果 $n$ 在 $\mathbb{F}_p$ 上已经有平方根,那么求出的结果一定也在 $\mathbb{F}_p$ 内,这是因为任意一个域中 $x^2 = n$ 都最多只有两个根(拉格朗日定理)。

\section{做法}


\begin{lemma*}[欧拉准则]
    $$
        x^{\frac{p-1}{2}} = 
            \begin{cases}
                1 & x\text{是二次剩余} \\
                -1 & x\text{不是二次剩余} \\
            \end{cases}
    $$
\end{lemma*}
首先考虑在$\mathbb{F}_p$中随机选取一个数$a$满足 $\omega = a^2 - n$ 不是二次剩余,由于$\mathbb{F}_p$ 有一半的数不是二次剩余,这一步很快,判定用欧拉准则即可。

现在考虑$\mathbb{F}_p$的二次扩张 $\mathbb{F}_p(\sqrt{\omega})$。

我们来证明$n$的一个平方根是:
$$
    (a+\sqrt{\omega})^{\frac{p+1}{2}}
$$

\begin{proof}
只需要证明 $(a+\sqrt{\omega})^{p+1} = n$ 即可:
\begin{align*}
    (a+\sqrt{\omega})^{p+1}
    &= (a+\sqrt{\omega})^{p}(a+\sqrt{\omega})\\
    &= (a^p+\sqrt{\omega}^p)(a+\sqrt{\omega}) &  \because 1< i<p \implies \binom{p}{i} = 0\\
    &= (a+\sqrt{\omega}^{p-1}\sqrt{\omega})(a+\sqrt{\omega}) \\
    &= (a+\omega^{\frac{p-1}{2}}\sqrt{\omega})(a+\sqrt{\omega}) \\
    &= (a-\sqrt{\omega})(a+\sqrt{\omega})&  \because \omega \text{不是二次剩余} \implies \omega^{\frac{p-1}{2}} = -1 \\
    &= a^2-\omega ^2 \\
    &= a^2-(a^2-n) \\
    &= n \\
\end{align*}
\end{proof}

所以 $(a+\sqrt{\omega})^{\frac{p+1}{2}}$确实是 $n$ 的平方根。如果 $n$ 在 $\mathbb{F}_p$ 中是二次剩余,那么我们得到的结果一定也在 $\mathbb{F}_p$中。






\end{document}
